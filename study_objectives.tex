\chapter{Study Objectives}
\section{Validating a Graph Theoretic Screening Approach to Food Item Combinations}

The aim of this study was to investigate and validate the concept that compatibility information on pairs of items, gathered via a consumer questioning procedure, can be used to predict larger combinations of items.  An important concept in this study was that this scaling up was tested at the individual level.  The validation of this scaling up at the individual level would give consumer researchers a new tool to investigate food combination consumption patterns.

\section{A Group Level Validation of the Supercombinatorality Property: Finding High-Quality Ingredient Combinations Using Pairwise Information}
The aim of this study was to extend the previous study by investigating the supercombinatorality property at the group level and with a new product system (pizza toppings).  This investigation showed techniques for combining individual pairwise response data into a group response matrix, which was then used to create predictions. These predictions were then checked by the same individuals.  If this study were successful it would show how to validate the supercombinatorality property with ones’ own product system, as well provide the first results of the type of supercombinatorality study that could be used in real-world scenarios.  
\section{A Graph Theoretic Approach to US Army Field Ration Menu Development}
Menus are unique types of combinations of foods because, unlike pizza and salad ingredients, the items to be combined come from separate categories.  This study extended the graph theoretic approach to menu development using the Meal-Ready-to-Eat™ and also introduced a ranking system for proposed combinations to predict their potential success.  Being able to predict menus from pairwise responses would allow researchers to overcome complex problems related to other menu development approaches and provide a comprehensive toolkit for investigating combinations of foods.