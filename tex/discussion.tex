\chapter{General Discussion and Conclusions}
\section{Discussion}
This research program provides the consumer science community with the first empirical investigation using graph theory to represent food connections.  It is believed that this methodology will be further developed and become a standard portion of the sensory scientists’ toolbox. 
The questioning procedure has subjects rate pairwise compatibility information on popular ingredients.  In the graph theoretic model this builds up the connectivity information necessary to build a graph (see Figures~\ref{fig:saladgraph} and~\ref{fig:pizzedge}).  Graphs are dynamic mathematical representations that have numerous analyses available for them.  The implementation of clique finding algorithms to explore larger combinations is only the tip of the iceburg of what can be done.  

The open challenge of finding optimal combinations of food components for various applications (MREs\tm, frozen meals, pizzas, salads, flavor combinations for yogurts/smoothies, etc.) led to the natural application of cliques to predict those combinations.  The usefulness of this approach was dependent on validation, however.  First, the salad study showed, for individuals, that pairwise information does provide good predictions of overall.  Table~\ref{tab:propsalad} illustrates that the majority of people chose predicted combinations over random combinations.  In fact, out of the 63 people in the study only 3 chose more non-cliques than cliques as being compatible.  This shows the potential of the method's remarkable abilities to create predictions.  

Even though the methodology was shown to hold at the individual level, it was important to provide methodology to combine individual group data together to form a group consensus (See Table~\ref{tab:pizztri}) and also show that, when these data were combined, that too much information was not lost.  Table~\ref{tab:pizzcompat} shows that in fact the scaling up does hold at the group level as well.  It is important to note, however, that these concepts were only validated for a pizza and salads and that, while probable, it is not a given that these results would be mirrored for all possible categories of foods.  Thus, Chapter IV also serves as a template for validating the method for other product categories.  Once validated, the method can be used faithfully to measure changes in perception over time, predict new combinations of foods, investigate competitors products and optimize product lines.

Chapter V introduced an extension to the experiment which served multiple purposes in this research program.  First and foremost we introduced methodology to extend the graph theoretic approach to menus – that is, optimizing items from different categories.  Second, by using MREs\tm, we chose an incredibly complex product with 11 different categories of food items and over 100 different components.  This stress-tested the method, as typical meals in other applications (schools, prisons, frozen meals, etc.) have 3 or 4 categories and only a handful of options in each category.  We also chose to focus on the entrée as the centerpiece of these meals and build our menus around the entrée frame-of-reference (Table~ref{tab:mremeals}).  This approach, combined with the ability to rank menus, provides extremely useful information to the product developer to help screen potential new menus. 

Some improvement to the methodology may help prevent against missing important combinations.  For example, by using a "clique-minus-1" technique one could find combinations that are almost cliques, but missing perhaps only one edge.  This would cast a wider net and provide the ability to catch more potential combinations, and it is conceivable that these combinations may have higher scores than cliques.  Another way to help prevent against missing important combinations would be to determine weighting for scores.  Currently, all category pairs are deemed equally important when attaching the score to the combinations in Chapter V.  If we knew that any pair containing the entr\'{e}e were more important, the score could be weighted accordingly, increasing the accuracy.

There are a number of obvious follow-up studies that should take place.  First, it is desired to take our predicted MRE\tm menus back to soldiers and run a validation experiment to test if our predictions fare better than random combinations.  We would also like to correlate this with individual liking information for the various components, as this would allow us to discover if there is a combination effect or not.  Second, these results should be replicated with additional product systems and with different groups of people.  Do children form the same complex pairwise relationships with foods?  Or are more simplistic regression based approaches good enough?  

\section{Conclusions}
These investigations have shown the potential for the graph theoretic approach to revolutionize the way combinations of foods are mathematically represented (from the analysts point of view) and developed (from the food product developer/ food scientist’s point of view).  The first two investigations have validated the necessary underlying assumptions needed for the method to be trusted.  The final investigation shows a complex real world application and demonstrates the type of results that can be expected when employing this methodology. 
